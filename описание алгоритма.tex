\documentclass[draft, 12pt]{article}
\usepackage[cp1251]{inputenc} % следующие две строки используются для
\usepackage[english,russian]{babel}   % руссификации AmSLaTeX
\usepackage{amsmath,amsfonts,amssymb,euscript,graphicx,wrapfig,multirow}
\usepackage{dsfont}
\textheight=240mm \textwidth=170mm
\hoffset=-17mm % сдвиг влево
\voffset=-17mm % сдвиг вверх

\begin{document}

\begin{center}
\textbf{О численном методе решения семейства интегро-дифференциальных уравнений с переменными коэффициентами, возникающих в финансовой математике} \\[3mm]
\textbf{О.~Е.~Кудрявцев}\\[2mm]
\emph{Ростовский филиал Российской таможенной академии}

\textbf{В.~В.~Родоченко}\\[2mm]
\emph{Южный Федеральный Университет}
\end{center}

\section{Аннотация}
Здесь представлен алгоритм вычисления и небольшое введение, позволяющее понять контекст и отладить основные шаги

\section{Предварительные соображения}
Система. Описывает поведение актива $S_t$. Процесс вариации $V_t$, подчиняется процессу CIR. 
\begin{equation}
\begin{array}{l}
dS_t = (r-\lambda_J \zeta)S_t dt+\sqrt{V_t} S_t dZ^S_t+(J-1) S_t dN_t,
\smallskip\\
dV_t= \kappa_V(\theta_V-V_t)dt+\sigma_V\sqrt{V_t}dZ^V_t,
\smallskip\\
<dZ^S_t,dZ^V_t> = \rho dt,
\smallskip\\
\end{array}
\end{equation}
где $r$ -- неотрицательный параметр, $Z^S_t$ и $Z^V_t$ - винеровские процессы, связанные коэффициентом корреляции $\rho$. В скачковой части, $N_t$ представляет из себя пуассоновский процесс с интенсивностью $\lambda_J$, считающий к моменту $t$ количество одинаково распределенных скачков размера $J$. Процесс $N_t$ не зависит от процессов $Z^S_t$ и $Z^V_t$, а также независим от $J$.  Параметр $\kappa_V$ определяет скорость ``возврата'' процесса вариации к ``долговременному'' среднему значению $\theta_V$, $\sigma_V>0$ называется ``волатильностью'' вариации. Соотношение между  $\zeta$ и параметрами распределения $J$ подбирается из соображения мартингальности процесса $\exp(-rt)S_t$. 

Величина скачков $J$ имеет логнормальное распределение $J\sim \mbox{LogN}(\mu_J, \sigma_J^2)$, тогда $\zeta = e^{\mu_J + \frac{1}{2}\sigma_J^2} - 1$

Математическое ожидание:
\begin{equation}
f(S,V,t) = M[e^{-r(T-t)}  {\bf 1}_{\underline{S}_T>H} G(S_T) | S_t = S, V_t = V],
\end{equation}
$H$ -- поглощающий барьер, $\underline{S}_T (= \inf_{0\le t\le T}S_T)$ -- процесс инфимума процесса $S_t$. 

Функция выплат (put)):
$G(S)=\max\{0, K - S\}$

Пусть $\tau = T-t$, тогда $F(S,V,\tau)\left(=f(S,V, T-\tau)\right)$ удовлетворяет следующему интегро-дифференциальному уравнению в частных производных с переменными коэффициентами в области $S(\tau) > H$.
\begin{eqnarray*}
		\frac{\partial F(S,V,\tau)}{\partial \tau} &= &
		\frac{1}{2} VS^2 \frac{\partial^2 F(S,V,\tau)}{\partial S^2} +
		\rho\sigma_V VS \frac{\partial^2 F(S,V,\tau)}{\partial S \partial V} +
		\\
		\frac{1}{2} \sigma_V^2 V \frac{\partial^2 F(S,V,\tau)}{\partial V^2} &+&
		(r - \lambda_J \zeta) S \frac{\partial F(S,V,\tau)}{\partial S} +
		\kappa_V(\theta_V - V) \frac{\partial F(S,V,\tau)}{\partial V} - 
		\\
		(r+\lambda_J)F(S,V,\tau) &+&
		\lambda_J \int_0^\infty F(JS,V,\tau) f(J) dJ,
		\\
		F(S,V,0) &=& G(S),
\end{eqnarray*}
где $f(J)$ -- функция плотности вероятностей величины скачков $J$. $F(S,V,\tau) = 0,  S(\tau) \le H$.

\section{Замена процесса}

Обозначим $\hat{\rho} = \sqrt{1-\rho^2}$, $W = Z^V$, $\rho W + \hat{\rho}Z = Z^S$, где $W_t$ $Z_t$ -- независимые броуновские движения. 

Введём замену для процесса $S_t$, положив $Y_t = \ln(\frac{S_t}{H}) - \frac{\rho}{\sigma_V}V_t$. При этом $S_t = H\exp(Y_t + \frac{\rho}{\sigma_V}V_t) $  Тогда:

\begin{equation*}
\begin{array}{l}
d Y_t = (r-\frac{1}{2}V_t - \frac{\rho}{\sigma_V}\kappa_V(\theta_V - V_t) - \lambda_J \zeta)dt+\hat{\rho}\sqrt{V_t} dZ_t+\ln JdN_t,
\smallskip\\
dV_t= \kappa_V(\theta_V-V_t)dt+\sigma_V\sqrt{V_t}dW_t.
\smallskip\\
\end{array}
\end{equation*}

Для дальнейшего описания алгоритма введём обозначения:

\begin{equation*}
\begin{array}{l}
\mu_Y(V_t) = r - \frac{1}{2}V_t - \frac{\rho}{\sigma_V}\kappa_V(\theta_V - V_t)  - \lambda_J \zeta,
\smallskip\\
\mu_V(V_t) = \kappa_V(\theta_V - V_t).
\smallskip\\
\end{array}
\end{equation*}

\section{Рандомизация Карра}
Для того, чтобы иметь возможность свести рассматриваемую задачу к задаче на малых интервалах времени, мы используем процедуру, известную как ``рандомизация Карра'', впервые введённую в статье [19] и обобщённую на общий случай задач стохастического управления в статье [20]. Обозначим $F_{n}({Y}_{t_n}, {V}_{t_n}) = F_{n}(H\exp(Y_t + \frac{\rho}{\sigma_V}V_t), {V}_{t_n}, t_n)$ -- приближение Карра значения функции $F(S,V,\tau)$ в момент времени $t_n$, где $t_i = i \Delta \tau$ и $\Delta \tau = \frac{T}{N}$. Пусть $\{\tau_i\}_{i=1}^N$ –- набор независимых экспоненциально распределённых случайных величин со средним $\Delta \tau$. Обозначим $Z_\tau^n=Y_{t_n + \tau}+\frac{\rho}{\sigma_V}V_{t_n + \tau}$.

Полагая $F_{N}(Y_T, V_T)=G(H\exp (Y_T+\frac{\rho}{\sigma_V}V_T ))$, получаем возможность записать:

$$
F_{n}({Y}_{t_n}, {V}_{t_n}) = M_{t_{n}} [ e^{-r\Delta \tau}  
I_{\underline{Z}^n_{\tau_n} > 0} F_{n+1}(Y_{t_n+\tau_n},V_{t_n+\tau_n})], \smallskip\\ n = N-1,..,0 
$$

\section{Аппроксимация}

Построим  биномиальное дерево со ``склеенными'' вершинами, определяемыми по формуле:
\begin{equation*}	
V(n,k) = (\sqrt{V_0} + \frac{\sigma_V}{2}(2k-n)\sqrt{\Delta \tau})^2 \mathds{1}_{(\sqrt{V_0} + \frac{\sigma_V}{2}(2k-n)\sqrt{\Delta \tau}) > 0}, 
\\ n = 0,1,...,N, \; k = 0,1,...,n.
\end{equation*}

Идея приближения состоит в том, что в каждый момент времени $t_n$ вариация может находиться в одном из состояний $V(n,k)$. В момент $t_{n+1}$ из вершины $(n,k)$ мы можем попасть либо ``вверх'' -- в вершину $(n+1,k_u)$, либо ``вниз'' -- в вершину $(n+1,k_d)$, при этом $k_u$ и $k_d$ подбираются так, чтобы согласовать движение по дереву со сносом $\mu(V_{(n,k)})$, по следующим правилам:
\begin{eqnarray*}
	k_u^{\Delta \tau}(n,k) = min\{{k^\ast} : k + 1 \le k^\ast \le n + 1, V(n,k) + \mu_{V}(V(n,k)){\Delta \tau} \le V(n+1,k^\ast)\} \\
	k_d^{\Delta \tau}(n,k) = max\{{k^\ast} : 0 \le k^\ast \le k, V(n,k) + \mu_{V}(V(n,k)) \ge V(n+1,k^\ast)\}
\end{eqnarray*}

Определим вероятности переходов как:

$$
p^{\Delta \tau}_{k^{\Delta \tau}_{u}(n,k)} = 
\frac{\mu_{V}(V(n,k))\Delta \tau+V(n,k)-V(n+1,k^{\Delta \tau}_{d}(n,k))}{V(n+1,k^{\Delta \tau}_{u}(n,k))-V(n+1,k^{\Delta \tau}_{d}(n,k))}
$$

Чтобы обеспечить корректную работу схемы в случае различных значений параметров, необходимо ввести дополнительные правила, предотвращающие появление отрицательных вероятностей в некоторых вершинах:
\begin{equation*}
p^{\Delta \tau}_{k^{\Delta \tau}_{u}(n,k)} :=
\begin{cases}
1, & p^{\Delta \tau}_{k^{\Delta \tau}_{u}(n,k)} > 1\\
p^{\Delta \tau}_{k^{\Delta \tau}_{u}(n,k)}, & p^{\Delta \tau}_{k^{\Delta \tau}_{u}(n,k)} \in [0, 1]\\
0, & p^{\Delta \tau}_{k^{\Delta \tau}_{u}(n,k)} < 0
\end{cases}
, \;\;\;\; p^{\Delta \tau}_{k^{\Delta \tau}_{d}(n,k)} := 1 - p^{\Delta \tau}_{k^{\Delta \tau}_{u}(n,k)},
\end{equation*}

\section{Приближённая факторизация}

Зафиксировав таким образом вариацию в каждом из узлов, мы имеем возможность рассматривать семейство задач с интегро-дифференциальным оператором следующего вида:

$$L_{n,k}f(y) := L_{Y}^{V(n,k)}f(y) = \frac{1}{2\pi}\int^\infty_{-\infty}e^{iy\xi}\psi_{n,k}(\xi)\hat{f}(\xi)d\xi.$$

Этот оператор можно понимать как псевдодифференциальный оператор, символом которого является $\psi_{n,k}(\xi)$ -- характеристическая экспонента процесса $Y_t$ при $V_t = V(n,k)$:

\begin{equation*}
	\psi_{n,k}(\xi) = \hat{\rho}^2\frac{V(n,k)}{2}\xi^2 - i\mu_Y(V(n,k))\xi + \phi(\xi),
\end{equation*}
где $\phi(\xi)$ -- характеристическая экспонента обобщённого пуассоновского процесса. Например, для модели Мертона она имеет вид: $\phi(\xi) = \lambda_J(1-e^{\frac{\sigma_J^2}{2} + \mu_J})$

Для каждого из узлов $(n,k), n=N-1,...0$ возникает две задачи -- одна в предположении, что переход был совершён в вершину $(n+1,k_d)$, другая -- в предположении, что переход был совершён в вершину $(n+1,k_u)$. Решение каждой из этих задач может быть записано в терминах операторов $\varepsilon^{+}_{q}$ и $\varepsilon^{-}_{q}$ -- факторов Винера-Хопфа (см статью [14]):

\begin{eqnarray*}
f_{n}^{k_{d}}(y) = (q{\Delta \tau})^{-1} \; \varepsilon^{-}_{q} \mathds{1}_{(- \frac{\rho}{\sigma_V}V(n,k), +\infty)}(y) \; \varepsilon^{+}_{q} f_{n+1}^{k_d}(y);\\
f_{n}^{k_{u}}(y) = (q{\Delta \tau})^{-1} \; \varepsilon^{-}_{q} \mathds{1}_{(- \frac{\rho}{\sigma_V}V(n,k), +\infty)}(y) \; \varepsilon^{+}_{q} f_{n+1}^{k_u}(y),
\end{eqnarray*}

Далее последовательно вычисляя $f_n^k = p_{k_d^{\Delta \tau}(n,k)} f_n^{k_d}(y) + p_{k_u^{\Delta \tau}(n,k)} f_n^{k_u}(y)$ для $n=N-1,..,0, k = 0,...,n,$
где $f_n^k = F(He^{y+\frac{\rho}{\sigma_V}V(n,k)}, V(n,k), n\Delta \tau)$, мы, после возвращения к исходным обозначениям, получаем приближённые значения искомого функционала (2). В отличие от более простого случая модели Хестона [21], наличие скачков лишает возможности использовать явные формулы для факторов -- их получить не удаётся. Аналитические формулы для факторов имеют вид:

\begin{eqnarray*}
	\phi^+_q(\xi)&=&\exp\left[(2\pi i)^{-1}
	\int_{-\infty+i\omega_-}^{+\infty+i\omega_-}\frac{\xi\ln(q+\psi(\eta))}
	{\eta(\xi-\eta)}d\eta\right];\\
	\phi^-_q(\xi)&=&\exp\left[-(2\pi i)^{-1}
	\int_{-\infty+i\omega_+}^{+\infty+i\omega_+}\frac{\xi\ln(q+\psi(\eta))}
	{\eta(\xi-\eta)}d\eta\right],
\end{eqnarray*}

Константы $\omega_+$ и $\omega_-$, такие, что $\omega_-<0<\omega_+$, имеют здесь смысл параметров и подбираются так, чтобы сохранить сходимость соответствующих интегралов и зависят от параметров процесса Леви.  В работе [18] получено универсальное и удобное для численной реализации представление факторов Винера-Хопфа. Функция $\phi^+_q(\xi)$ допускает аналитическое продолжение в полуплоскость $\Im \xi>\omega_-$ и может быть представлена как:
\begin{eqnarray*}
	\phi^+_q(\xi)&=&\exp\left[i\xi F^+(0)-\xi^2\hat{F}^+(\xi)\right],\\ 
	F^+(x)&=&\mathds{1}_{(-\infty,0]}(x)(2\pi)^{-1}
	\int_{-\infty+i\omega_-}^{+\infty+i\omega_-}e^{ix\eta}\frac{\ln(q+\psi(\eta))}
	{\eta^2}d\eta;
	\\
	\hat{F}^+(\xi)&=&\int_{-\infty}^{+\infty}e^{-ix\xi}F^+(x)dx.
\end{eqnarray*}

Аналогично, $\phi^-_q(\xi)$ допускает аналитическое продолжение в полуплоскость $\Im \xi<\omega_+$ и может быть представлена как:
\begin{eqnarray*}
	\phi^-_q(\xi)&=&\exp\left[-i\xi F^-(0)-\xi^2\hat{F}^-(\xi)\right],\\
	F^-(x)&=&\mathds{1}_{[0,+\infty)}(x)(2\pi)^{-1}
	\int_{-\infty+i\omega_+}^{+\infty+i\omega_+}e^{ix\eta}\frac{\ln(q+\psi(\eta))}
	{\eta^2}d\eta;\\
	 \hat{F}^-(\xi)&=&\int_{-\infty}^{+\infty}e^{-ix\xi}F^-(x)dx.
\end{eqnarray*}

Для вычисления интегралов используется алгоритм быстрого преобразования Фурье.



\end{document}
